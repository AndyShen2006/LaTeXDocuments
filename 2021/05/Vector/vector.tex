\documentclass[final,11pt,oneside,UTF8]{article}
\usepackage{ctex}
\usepackage{float}
\usepackage{geometry}
\usepackage{graphicx}
\usepackage{amsmath,amsfonts,amssymb}

\author{AndyShen2006}
\date{May.21st.2021}
\title{向量}
\begin{document}
\maketitle
\section{定义}
\subsection{向量的定义}
\paragraph{
    向量,即同时具有有方向的线段,图中表示为一个带箭头的线段
    而与向量相对应的量就是标量,标量只有大小,没有方向。
    正因为向量具有方向,所以在物理当中得到了广泛应用
    印刷体中向量通常印作黑色粗体,如 $\textbf{AB} $
    而书写的时候记作一个上面带个小箭头的一个量,如$\vec{AB}$
}
\subsection{*复数}
\paragraph{
    本节可以略过,此处只是提一下复数可以表达向量,对后面内容理解
    没有任何影响
}
\paragraph{
复数,表达形式有两种,一种是表达为实部虚部形式,另外一种是表达成
相角与模的形式。对应复数c写法分别是:$c=a+bi, c=e^{i\theta}$
}
\subsubsection{代数基本定理}
\paragraph{
    即一个n次方程在复数域内有n个解,证明此处不列
}
\subsubsection{棣莫弗公式}
\paragraph{
棣莫弗公式的表达为$e^{i\theta}=\cos{\theta}+i\sin{\theta}$\\
当$\theta=\pi$,即$\theta=180^{\circ}$时,则得到了著名的欧拉公式:
$$e^{i\pi}+1==0$$
关于棣莫弗公式的证明请自行查找,此处不再赘述
}
\subsection{矩阵}

\subsection{向量的表示}
\paragraph{
向量的表示一般有三种形式,即复数方式,矩阵方式,单位向量形式,
首先,我们需要了解,我们会将向量归一化,因为向量只有方向和大小,
所以因为平行线的的同位角相等,所以向量具有平移不变性。在实际使用
当中,我们就会将向量表达成一个以原点为起点
复数形式:
$$\textbf{V}=a+bi or \textbf{V}=a*e^{i\theta}$$
}
\subsection{向量的模}
\paragraph{
    前文中提到,向量对应的量就是标量,那么向量和标量之间
}
\end{document}