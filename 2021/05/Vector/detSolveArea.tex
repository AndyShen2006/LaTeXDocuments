\documentclass[final,11pt,oneside,UTF8]{article}
\usepackage{ctex}
\usepackage{float}
\usepackage{geometry}
\usepackage{graphicx}
\usepackage{amsmath,amsfonts,amssymb}

\author{AndyShen2006}
\date{May.21st.2021}
\title{行列式求面积的向量原理}
\begin{document}
\maketitle
\section{Introduction}
\paragraph{
    之前听父亲说过,说知道三边求面积用海伦公式,知道三点求面积用行列式。
    当时基本一脸懵,什么是行列式。而当父亲给我写出这个式子的时候,我惊
    到了,求面积竟然还有如此清新脱俗之法。在中学阶段,知道坐标求面积是
    非常困难且需要动用很多技巧的,但是行列式无需技巧,只要会计算就可以。
    但是每每我问起这个东西的数学原理时,得到的答案都是“这是一个巧合”。
    这真是一个巧合吗,在以前,我认为是,但当我仔细学习向量和行列式之后,
    我发现,这并不是一个巧合,我就写篇文章,简单说说这中间的原理。
}
\section{Problem}
\paragraph{
    假设已经有点A,B,C,坐标分别为(a1,a2),(b1,b2),(c1,c2),则存在该式:\\
    \[
        \pm 2S_{\triangle ABC}=
        \left|\begin{array}{cccc}
            a_1 & a_2 & 1 \\
            a_2 & b_2 & 1 \\
            c_1 & c_2 & 1
        \end{array}\right|
    \]
}
\section{Proof}
\subsection{向量的叉积}
\paragraph{
    本章默认认为已经读过了向量的基本表示和运算。向量的叉积一般有两种形式,
    设xyz轴的单位向量分别是$x_0 y_0 z_0, \textbf{A}=a_1x_0+a_2y_0+a_3z_0
        \textbf{B}=b_1x_0+b_2y_0+b_3z_0$则表达式分别是:
    1:行列式:\\
    \[
        \ \textbf{A} \times \textbf{B}=
        \left|\begin{array}{cccc}
            x_0 & y_0 & z_0 \\
            a_1 & a_2 & a_3 \\
            b_1 & b_2 & b_3
        \end{array}\right|
    \]
    2:三角函数:\\
    $$\textbf{A} \times \textbf{B} = n_0 \cdot |\textbf{A}| \cdot |\textbf{B}| \cdot \sin{\theta}$$
    仔细观察,有没有发现上面那个像我们行列式求面积的方式,下面那个像我们
    正弦求三角形面积的方式。这就是问题的关键所在。
}
\subsubsection{Main}
\paragraph{
    接着我们就说明这两者的相通性:\\
    很显然,这三个点在同一平面内,故z值全部取0。\\
    然后先将这三个向量归一化,我们以A点为原点,开始归一:\\
    $$\textbf{AB}=\textbf{B}-\textbf{A}=(b_1-a_1)x_0+(b_2-a_2)y_0$$
    $$\textbf{AC}=\textbf{C}-\textbf{A}=(c_1-a_1)x_0+(c_2-a_2)y_0$$
    很显然,我们由三角形面积公式,因为sin值可能是负数,所以得到:\\
    $$\pm 2S_{\triangle ABC}=AB \cdot AC \cdot \sin{\theta}$$
    此处标量AB,AC即为$AB=|\textbf{AB}|,AC=|\textbf{AC}|$
    然后我们将$\textbf{AB},\textbf{AC}$相乘,得到:
    $$\textbf{AB} \times \textbf{AC}
        =n_0 \cdot |\textbf{AB}| \cdot |\textbf{AC}| \cdot \sin{\theta}
    $$
    $$
        |\textbf{AB} \times \textbf{AC}|=AB \cdot AC \cdot \sin{\theta}=\pm 2S_{\triangle ABC}
    $$
    \[
        \ \textbf{AB} \times \textbf{AC}=
        \left|\begin{array}{cccc}
            x_0     & y_0     & z_0 \\
            b_1-a_1 & b_2-a_2 & 0   \\
            c_1-a_1 & c_2-a_2 & 0
        \end{array}\right|
    \]
    由余子式展开化简后得到
    \[
        \ |\textbf{AB} \times \textbf{AC}|=
        \left|\begin{array}{cccc}
            b_1-a_1 & b_2-a_2 \\
            c_1-a_1 & c_2-a_2
        \end{array}\right|=\pm 2S_{\triangle ABC}
    \]
    该式与下面一个式子等价:
    \[
        \pm 2S_{\triangle ABC}=
        \left|\begin{array}{cccc}
            0       & 0       & 1 \\
            b_1-a_1 & b_2-a_2 & 0 \\
            c_1-a_1 & c_2-a_2 & 0
        \end{array}\right|
    \]
    又与该式等价:
    \[
        \pm 2S_{\triangle ABC}=
        \left|\begin{array}{cccc}
            a_1 & a_2 & 1 \\
            a_2 & b_2 & 1 \\
            c_1 & c_2 & 1
        \end{array}\right|
    \]
    该式即为题目式子等价\\
    $\textbf{Q.E.D}$
}
\end{document}