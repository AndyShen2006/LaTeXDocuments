\documentclass[final,11pt,oneside,UTF8]{report}
\usepackage{ctex}
\usepackage{float}
\usepackage{geometry}
\usepackage{graphicx}
\usepackage{mhchem}
\usepackage{amsmath,amsfonts,amssymb}
\title{碱式碳酸铜制备时吸热现象的探究报告}
\author{AndyShen2006}
\date{Aug.10th.2021}
\begin{document}
\maketitle
\section{发现原因}
\paragraph{
    当我准备制备碱式碳酸铜时,我采取的试验方法不太标准,用了温度计作为搅拌棒,然后发现了一个奇妙的现象:
    温度计的温度从室温的$27^\circ C$变成了$15^\circ C$,温度竟然降了$12^\circ C$,理论上,硫酸铜
    属于酸式盐,碳酸氢钠属于碱式盐,两者反应属于酸碱反应,会放热,但是实验现象居然是吸热,这与常识不同,
    于是,我开始了理论分析。
}
\section{猜想}
\subsection{猜想1}
\paragraph{
    第一种猜想是我想到的,之前研究溶解热的时候是考虑溶解变为两种离子过程中的焓变,那么在这个反应当中,既然
    吸热肯定也会产生焓变,且焓变为正。
}
\paragraph{
    这里通俗的解释一下焓变。首先要有一个观念就是,能量就是物质,物质当中存在能量,所以一切物质都是有能量的(这句话很重要)。
    焓标准解释是在封闭体系中且没有体积功和其他功的情况下,就是体系中热力学能的变化。通俗的理解
    就是假设你有一个绝对隔热且能够保持气压的一个盒子,因为能量量肯定在这个盒子当中,那么在这个盒子中用热力学第一定律
    (能量不能凭空产生,也不能凭空消失)就能够知道总能量是不变的。对于这一个盒子当中的反应物,如果反应产生或者
    吸收热量,那这个能量必须从这个盒子当中获得或者释放到这个盒子当中,所以这个盒子失去或者吸收的能量就是焓变的量,
    焓变的单位是$kJ\cdot mol^{-1}$,焓变为正就是反应吸收了能量(盒子失去了正的能量),焓变为负就是反应释放了能量
    (盒子获得了能量,即失去了负的能量)。而反应物之前的能量总和就是反应物的生成焓$H_1$,反应后的能量总和就是产物的生成焓$H_2$。
    而因为总能量不变,存在公式$H_1+\Delta H=H_2$,此处的$\Delta H$即为焓变量。
}
\subsection{猜想2}
\paragraph{
    第二种猜想是我爸提出的,他认为,人喝可乐的时候会解暑,因为碳酸分解以及水蒸发会吸热,那么因为制备碱式碳酸铜的时候的确有
    二氧化碳产生,他认为是二氧化碳释放时带出了水使得水蒸发,从而吸热。这的确是一个也很合理的猜想,但是很容易就举出反例,例如
    用锌粒和硫酸反应产生氢气,用碳酸钙和盐酸反应产生二氧化碳,在这两个反应当中都会剧烈放热,而不是吸热,后面根据计算也更加否定
    了这个猜想。
}
\section{计算过程}
\paragraph{
    因为猜想2实在太不合理,所以我开始验证猜想1。我先假设碳酸氢钠为1mol,写出了反应的方程式:
    $$\ce{\frac{1}{2}CuSO4(aq) + NaHCO3(aq) -> \frac{1}{4}CuCO3(s) + \frac{1}{4}Cu(OH)2(s) + \frac{1}{2}Na2SO4(aq) + \frac{3}{4}CO2(g) + \frac{1}{4}H2O(l)}$$
    利用Wolfram Alpha$^{[2]}$查表得到所有生成焓,算出反应物的生成焓总和是
    $-1336kJ\cdot mol^{-1}$,产物的生成焓总和是$-1321.43kJ\cdot mol^{-1}$,反应的焓变是$15.19kJ\cdot mol^{-1}$
    焓变为正,说明反应的确吸热。
}
\paragraph{
    在实验当中,温度降了$12^\circ C$,根据公式$Q=cm\Delta t$可以得到,使得30g水温度降低$12^\circ C$需要的能量为$1.5048kJ$
    而我使用的碳酸氢钠为0.1mol,所以理论吸热是$1.519kJ\cdot mol^{-1}$,考虑各种误差之类,这个结果是合情合理的,故猜想1正确,该反应
    吸热。
}
\section{总结与反思}
\paragraph{
    这一次探究让我明白了一件事,不要忽略实验当中每一个细节,因为有可能当中一些细微的细节,就能够使得你有更多的发现,这一次我
    的发现也就是源于温度计的读数的异常从而探究。另外就是,要有探究和质疑的精神,例如我爸提出的猜想乍一想也有合理之处,但是探究
    之后会发现这其实并不合理。但我为了说明这不合理也进行了一番探究,让我又懂了很多东西。
}
\section{引用}
\paragraph{$^{[1]}$:参考自高等教育出版社《无机化学(第三版)》上册化学热力学一章}
\paragraph{$^{[2]}$:Wolfram Alpha,由Wolfram Research推出的一款科学搜索引擎,网址:www.wolframalpha.com}
\end{document}