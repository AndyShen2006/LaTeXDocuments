\documentclass[final,11pt,oneside,UTF8]{report}
\usepackage{ctex}
\usepackage{float}
\usepackage{geometry}
\usepackage{graphicx}
\usepackage{mhchem}
\usepackage{amsmath,amsfonts,amssymb}
\begin{document}
\section{Warning!!!本卷十分前半部分很搞人心态,没心理准备不要做}
\text{By AndyShen2006, Sept.20th.2021}
\subsection{正文}
\subsubsection{恶心人部分}
\paragraph{
    请写出或配平以下的化学方程式,如反应无法发生则打'x'
}
\paragraph{
    锌和氢氧化钠反应\rule[-3pt]{11cm}{0.001em}
}
\paragraph{
    配平:$\ce{P2I4 + P4 + H2O -> PH4I + H3PO4}$\rule[-3pt]{7cm}{0.001em}
}
\paragraph{
    请写出碳酸氢钠与硫酸铜生成碱式碳酸铜($\ce{Cu(OH)2.CuCO3}$)的反应方程式\\
    \rule[-3pt]{14cm}{0.001em}
}
\paragraph{
    配平:$\ce{P + NaOH + H2O -> NaHPO2 + PH3}$
    \rule[-3pt]{7cm}{0.001em}
}
\paragraph{
    配平:$\ce{K2Cr2O7 + KI + H2SO4 -> K2SO4 + Cr2(SO4)3 + I2 + H2O}$\\
    \rule[-3pt]{14cm}{0.001em}
}
\paragraph{
    写出氧化铅($\ce{PbO}$)和硫酸生成硫酸铅($\ce{PbSO4}$,难溶)的反应\\
    \rule[-3pt]{14cm}{0.001em}
}
\paragraph{
    写出铁和热的浓硫酸的反应:
    \rule[-3pt]{10cm}{0.001em}
}
\subsubsection{极其考验认真听讲及活学活用部分}
\paragraph{
    写出铁和冷的浓硫酸的反应:
    \rule[-3pt]{10cm}{0.001em}
}
\paragraph{
    请写出无水碳酸钠的吸水过程及在干燥空气当中的逆过程\\
    \rule[-3pt]{14cm}{0.001em}
}
\paragraph{
    请写出氢氧化钠和二氧化硅的反应\\
    \rule[-3pt]{14cm}{0.001em}\\
    由于同族的元素往往具有相似的性质,故请写出氢氧化铷($\ce{RbOH}$)与二氧化硅的反应\\
    \rule[-3pt]{14cm}{0.001em}\\
    继续探索:氢氧化钙和二氧化硅的反应\\
    \rule[-3pt]{14cm}{0.001em}
}
\paragraph{
    请写出氯化钙和硫酸反应的方程式\\
    \rule[-3pt]{14cm}{0.001em}
}
\paragraph{
    请写出碳酸钙和溴酸($\ce{HBr}$)的反应方程式\\
    \rule[-3pt]{14cm}{0.001em}
}
\paragraph{
    请写出氧化亚铁和硫酸的反应方程式\\
    \rule[-3pt]{14cm}{0.001em}
}
\paragraph{
    请写出氢氧化铁和盐酸的反应方程式(基础)\\
    \rule[-3pt]{14cm}{0.001em}
}
\paragraph{
    烧菜时会加一点醋加快补铁,,请写出反应方程式\\
    \rule[-3pt]{14cm}{0.001em}
}
\end{document}