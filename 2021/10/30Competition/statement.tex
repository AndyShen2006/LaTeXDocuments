\documentclass[final,11pt,oneside,UTF8]{report}
\usepackage{ctex}
\usepackage{float}
\usepackage{geometry}
\usepackage{graphicx}
\usepackage{mhchem}
\usepackage{amsmath,amsfonts,amssymb}
\usepackage{listings}
\usepackage{multicol}
\begin{document}
\centerline{\LARGE{普及组模拟试题}}
\centerline{}
\centerline{\LARGE{HZ-OI出题组}}
\centerline{}
\centerline{\LARGE{Oct.30th.2021}}
\centerline{}
\begin{table}[h]
    \centering
    \begin{tabular}{lllll}
        \hline
        \multicolumn{1}{|l|}{题目名称}       & \multicolumn{1}{l|}{吉布斯}    & \multicolumn{1}{l|}{小学数学题} & \multicolumn{1}{l|}{Constantinople} & \multicolumn{1}{l|}{成语接龙} \\ \hline
        \multicolumn{1}{|l|}{题目类型}       & \multicolumn{1}{l|}{传统型}    & \multicolumn{1}{l|}{传统型}     & \multicolumn{1}{l|}{传统型}         & \multicolumn{1}{l|}{传统型}   \\ \hline
        \multicolumn{1}{|l|}{目录}           & \multicolumn{1}{l|}{gibbs}     & \multicolumn{1}{l|}{math}       & \multicolumn{1}{l|}{ct}             & \multicolumn{1}{l|}{word}     \\ \hline
        \multicolumn{1}{|l|}{可执行文件名}   & \multicolumn{1}{l|}{gibbs}     & \multicolumn{1}{l|}{math}       & \multicolumn{1}{l|}{ct}             & \multicolumn{1}{l|}{word}     \\ \hline
        \multicolumn{1}{|l|}{输入文件名}     & \multicolumn{1}{l|}{gibbs.in}  & \multicolumn{1}{l|}{math.in}    & \multicolumn{1}{l|}{ct.in}          & \multicolumn{1}{l|}{word.in}  \\ \hline
        \multicolumn{1}{|l|}{输出文件名}     & \multicolumn{1}{l|}{gibbs.out} & \multicolumn{1}{l|}{math.out}   & \multicolumn{1}{l|}{ct.out}         & \multicolumn{1}{l|}{word.out} \\ \hline
        \multicolumn{1}{|l|}{每个测试点时限} & \multicolumn{1}{l|}{1.0秒}     & \multicolumn{1}{l|}{1.0秒}      & \multicolumn{1}{l|}{1.0秒}          & \multicolumn{1}{l|}{1.0秒}    \\ \hline
        \multicolumn{1}{|l|}{内存限制}       & \multicolumn{1}{l|}{512MiB}    & \multicolumn{1}{l|}{512MiB}     & \multicolumn{1}{l|}{512MiB}         & \multicolumn{1}{l|}{512MiB}   \\ \hline
        \multicolumn{1}{|l|}{子任务数目}     & \multicolumn{1}{l|}{5}         & \multicolumn{1}{l|}{10}         & \multicolumn{1}{l|}{10}             & \multicolumn{1}{l|}{5}        \\ \hline
        \multicolumn{1}{|l|}{测试点是否等分} & \multicolumn{1}{l|}{是}        & \multicolumn{1}{l|}{是}         & \multicolumn{1}{l|}{否}             & \multicolumn{1}{l|}{否}       \\ \hline
        提交源程序文件名                     &                                &                                 &                                     &                               \\ \hline
        \multicolumn{1}{|l|}{对于C++语言}    & \multicolumn{1}{l|}{gibbs.cpp} & \multicolumn{1}{l|}{math.cpp}   & \multicolumn{1}{l|}{ct.cpp}         & \multicolumn{1}{l|}{word.cpp} \\ \hline
        \multicolumn{1}{|l|}{对于C语言}      & \multicolumn{1}{l|}{gibbs.c}   & \multicolumn{1}{l|}{math.c}     & \multicolumn{1}{l|}{ct.c}           & \multicolumn{1}{l|}{word.c}   \\ \hline
        \multicolumn{1}{|l|}{对于Pascal语言} & \multicolumn{1}{l|}{gibbs.pas} & \multicolumn{1}{l|}{math.pas}   & \multicolumn{1}{l|}{ct.pas}         & \multicolumn{1}{l|}{word.pas} \\ \hline
        编译选项                             &                                &                                 &                                     &                               \\ \hline
        \multicolumn{1}{|l|}{对于C++语言}    & \multicolumn{1}{l}{}           & \multicolumn{1}{l}{-O2 -lm}     & \multicolumn{1}{l}{}                & \multicolumn{1}{l|}{}         \\ \hline
        \multicolumn{1}{|l|}{对于C语言}      & \multicolumn{1}{l}{}           & \multicolumn{1}{l}{-O2 -lm}     & \multicolumn{1}{l}{}                & \multicolumn{1}{l|}{}         \\ \hline
        \multicolumn{1}{|l|}{对于Pascal语言} & \multicolumn{1}{l}{}           & \multicolumn{1}{l}{-O2}         & \multicolumn{1}{l}{}                & \multicolumn{1}{l|}{}         \\ \hline
    \end{tabular}
\end{table}
\newpage

\centerline{\LARGE{$\textbf{吉布斯}\text{(gibbs)}$}}
\subsubsection{题目背景}
\paragraph{
    苦逼的syt又开始了他学习无机化学的一天……
}
\subsubsection{题目描述}
\paragraph{
    一天,syt小朋友在《无机化学》上看到了一个公式,叫做吉布斯─亥姆霍兹方程,是这样的:$\Delta G=\Delta H-T\Delta S$。
}
\paragraph{
    简单的来说啊,每个物质都有两个性质a和b,将生成物的a之和减去反应物的a之和得到的差减去生成物的b值之和与反应物的b值的之和乘上温度$T$的差的结果。这个结果表达成公式就是$(\Sigma a_\text{生成物}-\Sigma a_\text{反应物})-T\cdot (\Sigma b_\text{生成物}-\Sigma b_\text{反应物})$
    如果得到的值小于0,则反应自发,等于0反应平衡,大于0反应非自发。
}
\subsubsection{输入格式}
\paragraph{一行三个整数$n,m,T$,分别代表反应物数量,生成物数量,反应温度}
\paragraph{接下来n行,每行两个值,分别代表反应物的a和b}
\paragraph{接下来m行,每行两个值,分别代表生成物的a和b}
\subsubsection{输出格式}
\paragraph{一行一个字符串,如果能自发反应就输出"Yes",如果不能自发反应就输出"No",如果反应平衡,则输出"Equilibrium"}
\subsubsection{样例组}
\begin{table}[h]
    \begin{tabular}{|l|l|}
        \hline
        Input\#1:                                                                  & Output\#1: \\ \hline
        \begin{tabular}[c]{@{}l@{}}1 2 300\\ 3000 1\\ 2000 2\\ 2000 2\end{tabular} & No         \\ \hline
    \end{tabular}
\end{table}
\subsubsection{提示与说明}
\paragraph{对于$100\%$的数据,$1\leq n,m\leq 5$,均不超出double范围}
\newpage

\centerline{\LARGE{$\textbf{小学数学题}\text{(math)}$}}
\subsubsection{题目背景}
\paragraph{
    岳佬带着她的儿子来到的我们的教室。她的儿子正在做着他的数学老师布置给他做的数学题,但他的老师却绝没想到 $\text{Ta}$ 的学生的妈妈是一位语文老师,他妈妈所教的班里有这样一群数学天才(语文废柴)。
}
\subsubsection{题目描述}
\paragraph{
    岳佬的儿子不会做的一道数学题是这样的:给出了一个长度为 $n$ 的由数字组成的串,但你只需要保留其中的 $m$ 个数字,使这些数字按照原串中的相对顺序排列能得到一个最大的整数。请你帮他想想。
}
\subsubsection{输入格式}
\paragraph{输入共两行。}
\paragraph{第一行两个整数 $n,m$。}
\paragraph{第二行一个不含空格的长为 $n$ 的串,意义在题目描述中已给出。}
\subsubsection{输出格式}
\paragraph{一行一个整数表示结果。}
\subsubsection{样例组}
\begin{table}[h]
    \begin{tabular}{|l|l|}
        \hline
        Input\#1:                                               & Output\#1: \\ \hline
        \begin{tabular}[c]{@{}l@{}}9 6\\ 597382064\end{tabular} & 982064     \\ \hline
    \end{tabular}
\end{table}
\subsubsection{提示与说明}
\paragraph{对于 $10\%$ 的数据,$n\le 20$。}
\paragraph{对于 $50\%$ 的数据,$n\le 10^4$。}
\paragraph{对于 $100\%$ 的数据,$m\le n\le 10^5$。}
\newpage

\centerline{\LARGE{$\textbf{Constantinople}\text{(ct)}$}}
\subsubsection{题目背景}
\paragraph{
    $1453$ 年,奥斯曼帝国大举进攻君士坦丁堡。而神仙
    $\text{zjy}$ 并不想看到君堡沦陷,于是他发动神技,使君堡的守军增多了。
}
\subsubsection{题目描述}
\paragraph{
    奥斯曼帝国有 $n$ 支军队进攻君士坦丁堡,对于每支军队,$\text{ zjy }$应该在 $d_i$ 时刻前将其歼灭。但如果一支奥斯曼军队没有在该时刻前被歼灭,这支军队就会攻入君堡进行掠夺,造成 $w_i$ 点损失。$\text{ zjy }$的$\text{ CPU0 }$使他一次只能迎击一支奥斯曼军队,并花费一单位时间。他现在想知道,奥斯曼军队最少会对君士坦丁堡造成多少点损失。
}
\subsubsection{输入格式}
\paragraph{
    第一行一个正整数 $n$,表示奥斯曼军队的数量。
}
\paragraph{
    接下来 $n$ 行,每行一个非负整数 $d_i$,$w_i$ 。
}
\subsubsection{输出格式}
\paragraph{
    一行一个整数,表示奥斯曼军队最少会对君士坦丁堡造成多少损失。
}
\subsubsection{样例组}

\begin{table}[h]
    \begin{tabular}{|l|l|}
        \hline
        Input\#1:                                                                           & Output\#1: \\ \hline
        \begin{tabular}[c]{@{}l@{}}6\\ 5 75\\ 3 24\\ 3 699\\ 1 56\\ 1 8\\ 5 17\end{tabular} & 8          \\ \hline
    \end{tabular}
\end{table}
\begin{table}[h]
    \begin{tabular}{|l|l|}
        \hline
        Input\#2:                                                                                                                                                    & Output\#2: \\ \hline
        \begin{tabular}[c]{@{}l@{}}15\\ 1 36\\ 2 89\\ 2 78\\ 2 44\\ 5 7\\ 6 10\\ 4 15\\ 3 12\\ 4 114514\\ 6 11\\ 6 13\\ 2 19\\ 4 1919\\ 5 2243\\ 6 2214\end{tabular} & 167        \\ \hline
    \end{tabular}
\end{table}
\subsubsection{提示与说明}
\paragraph{
    对于 $100\%$ 的数据,$1\le n\le 10000$,保证所有的 $d_i$,$w_i$ 都在 $int$ 范围内。
}
\newpage

\centerline{\LARGE{$\textbf{成语接龙}\text{(word)}$}}
\subsubsection{题目背景}
\paragraph{人除我佬,此题欢乐 AK。 ——syt}
\subsubsection{题目描述}
\paragraph{一天,$\text{hpc}$ 小朋友找到了 $\text{syt}$ 小朋友玩成语接龙,说出了四个字“三顾茅庐”,若干轮后,进入了死局“为所欲为”。好胜心强的 $\text{syt}$ 小朋友感到非常愤怒,想要写一个程序知道是成语如何变成为所欲为的。由于$\text{syt}$ 小朋友实在太蒻了,所以他请了 $\text{AK}$ 了 $10^9$ 次 $\text{IOI}$ 的你解决这个问题。他给了你一本成语词典和开始成语,问你到达结束成语的全过程和共计步数(指每次转换的步数)。}
\subsubsection{输入格式}
\paragraph{第 $1$ 行一个字符串 $start$ 代表起始成语。}
\paragraph{第 $2$ 行一个字符串 $end$ 代表结束成语。}
\paragraph{第 $3$ 行一个数字 $n$ 代表字典大小。}
\paragraph{第 $4$ 到 $n+3$ 行,每行一个字符串 $s$,代表字典中的一个成语。}
\subsubsection{输出格式}
\paragraph{输出第一行一个整数,代表转换次数。}
\paragraph{第二行一个串,代表转换过程,使用“`->`”作为连接,“`->`”前后一个空格。若不存在路径,输出 $-1$。}
\subsubsection{样例组}
\begin{table}[h]
    \begin{tabular}{|l|l|}
        \hline
        Input\#1:                                                                    &
        Output\#1:                                                                                                \\ \hline
        \begin{tabular}[c]{@{}l@{}}ABCD\\ EFBA\\ 3\\ DEFG\\ GADE\\ EFBA\end{tabular} &
        \begin{tabular}[c]{@{}l@{}}3\\ ABCD -\textgreater DEFG -\textgreater GADE -\textgreater EFBA\end{tabular} \\ \hline
    \end{tabular}
\end{table}
\newpage
\subsubsection{提示与说明}
\paragraph{样例中,转换过程是从 $ABCD$ 到 $DEFG$ 到 $GADE$ 到 $EFBA$,中间转换了 $3$ 次(就是箭头数)。}
\paragraph{对于 $10\%$ 的数据满足 $n=1$}
\paragraph{对于 $25\%$ 的数据满足 $1\leq n \leq 100$,且串长度不超过 $4$。}
\paragraph{对于 $50\%$ 的数据满足 $1\leq n\leq 1000$。}
\paragraph{对于 $100\%$ 的数据满足 $1\leq n\leq 1\times 10^5$ 且串长度均不超过 $6$,保证最短路径只有一条,字典 **一定** 包括字符串 $end$,且全部由大写字母组成。}
\paragraph{本题采用 $\text{subtask}$,必须同一 $\text{subtask}$ 下所有数据通过才能得到该 $\text{subtask}$ 的分数,且数据已加强。}
\newpage
\end{document}