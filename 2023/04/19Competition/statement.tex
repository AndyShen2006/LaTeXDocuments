\documentclass[final,11pt,oneside,UTF8]{report}
\usepackage{ctex}
\usepackage{float}
\usepackage{geometry}
\usepackage{graphicx}
\usepackage{mhchem}
\usepackage{amsmath,amsfonts,amssymb}
\usepackage{listings}
\usepackage{multicol}
\usepackage{multirow}
\usepackage{setspace}
\begin{document}
\centerline{\LARGE{杭州四中模拟赛试题}}
\centerline{}
\centerline{\LARGE{比赛时间:2023年4月19日18:00-20:30}}
\centerline{}
\centerline{出题人:杭州第四中学\ 沈奕天}
\centerline{}
\begin{table}[H]
    \centering
    \begin{tabular}{ccccc}
    \hline
    \multicolumn{1}{|c|}{题目名称}    & \multicolumn{1}{c|}{翻转}       & \multicolumn{1}{c|}{老刘的队伍}   & \multicolumn{1}{c|}{跳格子}      & \multicolumn{1}{c|}{价值}        \\ \hline
    \multicolumn{1}{|c|}{题目类型}    & \multicolumn{1}{c|}{传统型}      & \multicolumn{1}{c|}{传统型}     & \multicolumn{1}{c|}{传统型}      & \multicolumn{1}{c|}{传统型}       \\ \hline
    \multicolumn{1}{|c|}{目录}      & \multicolumn{1}{c|}{nega}     & \multicolumn{1}{c|}{seq}     & \multicolumn{1}{c|}{jump}     & \multicolumn{1}{c|}{value}     \\ \hline
    \multicolumn{1}{|c|}{可执行文件名}  & \multicolumn{1}{c|}{nega}     & \multicolumn{1}{c|}{seq}     & \multicolumn{1}{c|}{jump}     & \multicolumn{1}{c|}{value}     \\ \hline
    \multicolumn{1}{|c|}{输入文件名}   & \multicolumn{1}{c|}{nega.in}  & \multicolumn{1}{c|}{seq.in}  & \multicolumn{1}{c|}{jump.in}  & \multicolumn{1}{c|}{value.in}  \\ \hline
    \multicolumn{1}{|c|}{输出文件名}   & \multicolumn{1}{c|}{nega.out} & \multicolumn{1}{c|}{seq.out} & \multicolumn{1}{c|}{jump.out} & \multicolumn{1}{c|}{value.out} \\ \hline
    \multicolumn{1}{|c|}{每个测试点时限} & \multicolumn{1}{c|}{1.0秒}     & \multicolumn{1}{c|}{1.0秒}    & \multicolumn{1}{c|}{1.0秒}     & \multicolumn{1}{c|}{1.0秒}      \\ \hline
    \multicolumn{1}{|c|}{内存限制}    & \multicolumn{1}{c|}{512MiB}   & \multicolumn{1}{c|}{512MiB}  & \multicolumn{1}{c|}{512MiB}   & \multicolumn{1}{c|}{512MiB}    \\ \hline
    \multicolumn{1}{|c|}{子任务数目}   & \multicolumn{1}{c|}{20}       & \multicolumn{1}{c|}{20}      & \multicolumn{1}{c|}{20}       & \multicolumn{1}{c|}{20}        \\ \hline
    \multicolumn{1}{|c|}{测试点是否等分} & \multicolumn{1}{c|}{是}        & \multicolumn{1}{c|}{是}       & \multicolumn{1}{c|}{是}        & \multicolumn{1}{c|}{是}         \\ \hline
    \multicolumn{1}{l}{提交源程序文件名}  &                               &                              &                               &                                \\ \hline
    \multicolumn{1}{|c|}{对于C++语言} & \multicolumn{1}{c|}{nega.cpp} & \multicolumn{1}{c|}{seq.cpp} & \multicolumn{1}{c|}{jump.cpp} & \multicolumn{1}{c|}{value.cpp} \\ \hline
    编译选项                          &                               &                              &                               &                                \\ \hline
    \multicolumn{1}{|l|}{对于C++语言} & \multicolumn{4}{c|}{-O2 -std=c++14 -static}                                                                                   \\ \hline
    \end{tabular}
\end{table}
\begin{spacing}{0}
    \subsubsection{注意事项(请仔细阅读)}
    \paragraph{1.文件名(程序名和输入输出文件名)必须使用英文小写。}
    \paragraph{2.C/C++中函数main()的返回值类型必须是int,程序正常结束是的返回值必须是0。}
    \paragraph{3.提交的程序代码文件的放置位置请参考各省的具体要求。}
    \paragraph{4.因违反以上三点而出现的错误或问题,申诉时一律不予受理。}
    \paragraph{5.若无特殊说明,结果的比较方式为全文比较(过滤行末空格及文末回车)。}
    \paragraph{5.选手提交的程序源文件大小必须不大于100KB。}
    \paragraph{7.程序可使用的栈空间内存限制与题目的内存限制一致}
    \paragraph{8.统一评测时采用的机器配置为:Intel(R)Core(TM)i5-9500 CPU@3.00GHz,
        内存8GB。上述时限以此配置为准}
    \paragraph{9.只提供Linux格式附加样例文件}
    \paragraph{10.评测在Windows10 专业版(LemonLime 0.3.4.2)下进行,C++编译器版本为}
    \paragraph{}
\end{spacing}
\newpage
\centerline{\LARGE{$\textbf{翻转}\text{(nega)}$}}
\subsubsection{题目描述}
\paragraph{
    小A在玩一个翻牌子的游戏,这个游戏的规则是这样的:小C给小A一个数$x$,然后小A将它手中的序列进行$x^3+3x^2-x+1$次翻转(翻转即为将序列中所有0变成1,所有1变成0),然后输出翻转后的结果,请输出这个结果。
}
\subsubsection{输入格式}
\paragraph{
    第一行一个整数$x$,代表小A给出的这个数。
}
\paragraph{
    第二行一个01序列s,代表初始的序列。
}
\subsubsection{输出格式}
\paragraph{
    输出一行一个序列,代表翻转后的序列。
}
\subsubsection{样例组}
\begin{table}[H]
    \begin{tabular}{|l|l|}
        \hline
        Input \#1                                         & Output \#1 \\ \hline
        \begin{tabular}[c]{@{}l@{}}4\\ 10101\end{tabular} & 01010      \\ \hline
    \end{tabular}
\end{table}
\subsubsection{提示与说明}
\paragraph{
    对于$100\%$的数据,保证$1 \leq x \leq 1\times 10^6$,$100 \leq length(s) \leq 1\times 10^6$
}
\newpage
\centerline{\LARGE{$\textbf{老刘的队伍}\text{(seq)}$}}
\subsubsection{题目背景}
\paragraph{
    期中考临近,同学们或多或少有些问题需要问老师,当刘老师看着他长长的答疑队伍时,他陷入了沉思……
}
\subsubsection{题目描述}
\paragraph{
    刘老师为了尽可能提高班级的成绩,所以他请他亲爱的课代表给队伍中的人打了一个重要度。
    由于刘老师眼睛不太好,所以他每次只能够看到队列的前$k$个人,他想知道这$k$个人重要度最大值与最小值之差是多少。
}
\paragraph{
    显然刘老师回答完前一个人的问题后这个人就会离开,此时队列会发生变化,刘老师希望你编写一个程序,求出每次变化后新的差值序列的内容并输出。
}
\subsubsection{输入格式}
\paragraph{
    第一行两个整数$n,k$,$n$代表队伍中的总人数,$k$意义在题设中已给出。
}
\paragraph{
    第二行$n$个整数$v_i$,代表这些人的重要度。
}
\subsubsection{输出格式}
\paragraph{
    输出一行,表示全部的差值序列。
}
\subsubsection{样例组}
\begin{table}[H]
    \begin{tabular}{|l|l|}
    \hline
    Input \#1                                                       & Output \#1  \\ \hline
    \begin{tabular}[c]{@{}l@{}}8 3\\ 1 3 -1 -3 5 3 6 7\end{tabular} & 4 6 8 8 3 4 \\ \hline
    \end{tabular}
\end{table}
\subsubsection{提示与说明}
\paragraph{
    对于$50\%$的数据,$1\leq n\leq 10^3$。
}
\paragraph{
    对于$100\%$的数据,$1\leq k \leq n \leq 10^6$,所有数保证在long long范围以内。
}
\newpage
\centerline{\LARGE{$\textbf{跳格子}\text{(jump)}$}}
\subsubsection{题目描述}
\paragraph{
    小A和小C在玩一个跳格子的游戏,规则是这样的:我们先给出一个定点作为原点,
    他们一开始站在一个环的两个位置$x,y$,小A每次能跳$m$格,小C每次能跳$n$格,
    这个环的长度为$L$。小A和小C跳跃的次数于方向均相同,问他们跳多少次之后能够相见。
}
\subsubsection{输入格式}
\paragraph{
    一行五个整数$x,y,m,n,L$,意义在题目中已给出。
}
\subsubsection{输出格式}
\paragraph{
    一行一个整数,表示跳跃的次数。如果无法相遇,请输出`Impossible'
}
\subsubsection{样例组}
\begin{table}[H]
    \begin{tabular}{|l|l|}
    \hline
    Input \#1 & Output \#1 \\ \hline
    1 2 3 4 5 & 4          \\ \hline
    \end{tabular}
    \end{table}
\subsubsection{提示与说明}
\paragraph{
    对于100\%的数据,$1\leq x\neq y \leq 2\times 10^9,1\leq m,n\leq 2\times 10^9,1\leq L \leq 2.1\times 10^9$
}
\newpage
\centerline{\LARGE{$\textbf{价值}\text{(value)}$}}
\subsubsection{题目描述}
\paragraph{
    给定一序列$a_1,a_2,\cdots,a_n$,有若干个询问,每个询问包含一个区间$[l,r]$和一个值$k$,请回答该区间内第$k$大的值。
}
\subsubsection{输入格式}
\paragraph{
    第一行两个整数$n,q$,代表序列长度和询问次数。
}
\paragraph{
    第二行一个序列$a_1,a_2,\cdots,a_n$,代表这个序列的值。
}
\paragraph{
    接下来$q$行,每行三个整数$l,r,k$,意义在题目中已给出。
}
\subsubsection{输出格式}
\paragraph{
    输出$q$行,代表这些询问的答案。
}
\subsubsection{样例组}
\begin{table}[H]
    \begin{tabular}{|l|l|}
    \hline
    Input \#1                                                                                                           & Output \#1                                                                   \\ \hline
    \begin{tabular}[c]{@{}l@{}}5 5\\ 25957 6405 15770 26287 26465 \\ 2 2 1\\ 3 4 1\\ 4 5 1\\ 1 2 2\\ 4 4 1\end{tabular} & \begin{tabular}[c]{@{}l@{}}6405\\ 15770\\ 26287\\ 25957\\ 26287\end{tabular} \\ \hline
    \end{tabular}
    \end{table}
\subsubsection{提示与说明}
\paragraph{
    对于$30\%$的数据,保证$q=1,1\leq n\leq 1000$
}
\paragraph{
    对于另$30\%$的数据,保证$k=1$。
}
\paragraph{
    对于$100\%$的数据,满足$1\leq n,q \leq 2\times 10^5,|a_i|\leq 10^9,1\leq l \leq r \leq n, 1\leq k \leq r-l+1$
}
\newpage
\end{document}
