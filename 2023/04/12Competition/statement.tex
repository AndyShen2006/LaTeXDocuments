\documentclass[final,11pt,oneside,UTF8]{report}
\usepackage{ctex}
\usepackage{float}
\usepackage{geometry}
\usepackage{graphicx}
\usepackage{mhchem}
\usepackage{amsmath,amsfonts,amssymb}
\usepackage{listings}
\usepackage{multicol}
\usepackage{multirow}
\usepackage{setspace}
\begin{document}
\centerline{\LARGE{提高组模拟试题(简单于提高)}}
\centerline{}
\centerline{\LARGE{Apr.12th.2023}}
\centerline{}
\centerline{出题人:沈奕天}
\centerline{}
\begin{table}[h]
    \centering
    \begin{tabular}{lllll}
        \hline
        \multicolumn{1}{|l|}{题目名称}    & \multicolumn{1}{l|}{算账}        & \multicolumn{1}{l|}{大米老师}                  & \multicolumn{1}{l|}{金钱游戏}      \\ \hline
        \multicolumn{1}{|l|}{题目类型}    & \multicolumn{1}{l|}{传统型}       & \multicolumn{1}{l|}{传统型}                   & \multicolumn{1}{l|}{传统型}       \\ \hline
        \multicolumn{1}{|l|}{目录}      & \multicolumn{1}{l|}{count}     & \multicolumn{1}{l|}{dami}                  & \multicolumn{1}{l|}{money}     \\ \hline
        \multicolumn{1}{|l|}{可执行文件名}  & \multicolumn{1}{l|}{count}     & \multicolumn{1}{l|}{dami}                  & \multicolumn{1}{l|}{money}     \\ \hline
        \multicolumn{1}{|l|}{输入文件名}   & \multicolumn{1}{l|}{count.in}  & \multicolumn{1}{l|}{dami.in}               & \multicolumn{1}{l|}{money.in}  \\ \hline
        \multicolumn{1}{|l|}{输出文件名}   & \multicolumn{1}{l|}{count.out} & \multicolumn{1}{l|}{dami.out}              & \multicolumn{1}{l|}{money.out} \\ \hline
        \multicolumn{1}{|l|}{每个测试点时限} & \multicolumn{1}{l|}{1.0秒}      & \multicolumn{1}{l|}{1.0秒}                  & \multicolumn{1}{l|}{1.0秒}      \\ \hline
        \multicolumn{1}{|l|}{内存限制}    & \multicolumn{1}{l|}{512MiB}    & \multicolumn{1}{l|}{512MiB}                & \multicolumn{1}{l|}{512MiB}    \\ \hline
        \multicolumn{1}{|l|}{子任务数目}   & \multicolumn{1}{l|}{10}        & \multicolumn{1}{l|}{10}                    & \multicolumn{1}{l|}{10}        \\ \hline
        \multicolumn{1}{|l|}{测试点是否等分} & \multicolumn{1}{l|}{是}         & \multicolumn{1}{l|}{是}                     & \multicolumn{1}{l|}{是}         \\ \hline
        提交源程序文件名                      &                                &                                            &                                \\ \hline
        \multicolumn{1}{|l|}{对于C++语言} & \multicolumn{1}{l|}{count.cpp} & \multicolumn{1}{l|}{dami.cpp}              & \multicolumn{1}{l|}{money.cpp} \\ \hline
        编译选项                          &                                &                                            &                                \\ \hline
        \multicolumn{1}{|l|}{对于C++语言} & \multicolumn{1}{l}{}           & \multicolumn{1}{l}{-O2 -std=c++14 -static} & \multicolumn{1}{l|}{}          \\ \hline
    \end{tabular}
\end{table}
\begin{spacing}{0}
    \subsubsection{注意事项(请仔细阅读)}
    \paragraph{1.文件名(程序名和输入输出文件名)必须使用英文小写。}
    \paragraph{2.C/C++中函数main()的返回值类型必须是int,程序正常结束是的返回值必须是0。}
    \paragraph{3.提交的程序代码文件的放置位置请参考各省的具体要求。}
    \paragraph{4.因违反以上三点而出现的错误或问题,申诉时一律不予受理。}
    \paragraph{5.若无特殊说明,结果的比较方式为全文比较(过滤行末空格及文末回车)。}
    \paragraph{5.选手提交的程序源文件大小必须不大于100KB。}
    \paragraph{7.程序可使用的栈空间内存限制与题目的内存限制一致}
    \paragraph{8.全国统一评测时采用的机器配置为:Intel(R)Core(TM)i7-8700K CPU@3.70GHz,
        内存32GB。上述时限以此配置为准}
    \paragraph{9.只提供Linux格式附加样例文件}
    \paragraph{10.评测在当前最新公布的NOI Linux下进行,各语言的编译器版本以此为准}
    \paragraph{}
\end{spacing}
\newpage
\centerline{\LARGE{$\textbf{算账}\text{(count)}$}}
\subsubsection{题目背景}
\paragraph{
    小Z借了小J若干次钱,由于小J借了太多次钱,所以小Z要将借出去的前连本带息的收回来
}
\subsubsection{题目描述}
\paragraph{
    小Z是这么算账的:首先,小J要将借的钱全部还回;然后,小J要额外交还给他利息,利息计算方式是这样:
}
\paragraph{
    将所有借的钱按照金额从小到大排序,然后最大值减去最小值,次大值减去次小值……以此类推,得到的所有值的和即为利息,如果是奇数笔钱,显然会多一笔,
    那么慷慨的小Z就不收多出来这笔钱的利息。
}
\subsubsection{输入格式}
\paragraph{第一行一个整数 $n$,代表了有多少笔钱}
\paragraph{第二行$n$个整数,代表了每笔借的钱的金额}
\subsubsection{输出格式}
\paragraph{一行一个整数表示小Z应该收回的金额}
\subsubsection{样例组}
\begin{table}[H]
    \begin{tabular}{|l|l|}
        \hline
        Input\#1: & Output\#1: \\ \hline
        5         & 4          \\
        1 4 2 3 5 & 21         \\ \hline
    \end{tabular}
\end{table}
\subsubsection{提示与说明}
\paragraph{原序列排序后是$1,2,3,4,5$,金额是$((5-1)+(4-2))+(1+2+3+4+5)=21$}
\paragraph{对于$100\%$的数据,$1 \le n \le 10^6$}
\newpage

\centerline{\LARGE{$\textbf{大米老师}\text{(dami)}$}}
\subsubsection{题目描述}
\paragraph{
    小G加入了从$0$到$n$标号的$n+1$个QQ群,大米老师在第0个QQ群
}
\paragraph{
    每天,小G的朋友会给标号为$1$到$n$的群当中的一部分发信息,然后小G会筛选这些信息当中大米老师喜欢的并转发它们到
    第$0$个群
}
\paragraph{
    现在我们定义一个消息为一个只包含小写字母的字符串,然后大米老师喜欢的消息是字符串中存在一个子串为“bie”的消息。
}
\paragraph{
    现在,给出$1$到$n$中群中的消息,小G会按$1$到$n$的顺序寻找信息,并且一个个确认这些群中的每一条消息。对于每个消息,
    找出大米老师是否喜欢这条消息并且这条消息未在群$0$中出现过,然后将这条消息转发到群$0$。这里的出现指的是相同
    的消息在之前已经被转发到组$0$过
}
\paragraph{
    请按照输入顺序输出所有小G会转发到组0的消息。对于每一个群,如果小G不能找出符合要求的消息,输出一行为"Time to play Genshin
    Impact, Teacher Rice!"
}
\subsubsection{输入格式}
\paragraph{
    第一行包含了一个整数$n$,表示QQ群的个数,接下来的几行表示从群$1$到群$n$的所有消息。
}
\paragraph{
    对于第$i$个群,第一行包含了一个整数$m_i$,表示在这个群中有的消息数。接下来$m_i$行,每行一个字符串$s_{i,j}$,代表第
    $i$个群中第$j$条消息。
}
\subsubsection{输出格式}
\paragraph{
    按照输入顺序每行输出一条消息,代表小G会转发到群里的消息。对于每个群,如果小G无法找出任何符合条件的消息,那么输出
    一行为“Time to play Genshin Impact, Teacher Rice!”
}
\subsubsection{样例组}
\begin{table}[H]
    \begin{tabular}{|l|l|}
        \hline
        Input\#1                                                                                                                                                                   & Output\#1                                                                                                                                                                                                                 \\ \hline
        \begin{tabular}[c]{@{}l@{}}6\\ 1\\ biebie\\ 1\\ adwlknafdoaihfawofd\\ 3\\ ap\\ ql\\ biebie\\ 2\\ pbpbpbpbpbpbpbpb\\ bbbbbbbbbbie\\ 0\\ 3\\ abie\\ bbie\\ cbie\end{tabular} & \begin{tabular}[c]{@{}l@{}}biebie\\ Time to play Genshin Impact, Teacher Rice!\\ Time to play Genshin Impact, Teacher Rice!\\ bbbbbbbbbbie\\ Time to play Genshin Impact, Teacher Rice!\\ abie\\ bbie\\ cbie\end{tabular} \\ \hline
    \end{tabular}
\end{table}
\subsubsection{提示与说明}
\paragraph{
    对于$100\%$的数据,满足$1 \le n \le 10^4, 0\leq m_i \leq 10^4, \sum m_i \leq 10^5$
}
\newpage
\centerline{\LARGE{$\textbf{金钱游戏}\text{(money)}$}}
\subsubsection{题目描述}
\paragraph{
    小P和小B在和$n$个玩家进行一个游戏,每个玩家有一些存款,这些存款一定为一个实数。玩家$i$一开始有$a_i$的
    存款,在每一轮游戏中,游戏以如下方式有序进行:
}
\paragraph{
    玩家$1$给玩家$2$一半玩家$1$的存款。
}
\paragraph{
    玩家$2$给玩家$3$一半玩家$2$的存款。
}
\paragraph{
    $\cdots$
}
\paragraph{
    玩家$n-1$给玩家$n$一半玩家$n$的存款。
}
\paragraph{
    玩家$n$给玩家$1$一半玩家$n$的存款。
}
\paragraph{
    这$n$个玩家会玩这个游戏$2023^{4110}$次,小P好奇游戏后所有人手上剩下多少存款,请你编写一个程序回答这个问题。
}
\subsubsection{输入格式}
\paragraph{
    第一行包含一个整数$n$,表示游戏的人数。
}
\paragraph{
    第二行包含$n$个战术$a_1,a_2,\cdots,a_n$,表示每个人手上初始的存款。
}
\subsubsection{输出格式}
\paragraph{
    输出一行实数,代表游戏后每个人手上的存款。
}
\paragraph{
    注意,答案保留两位小数即可!
}
\subsubsection{样例组}
\begin{table}[H]
    \begin{tabular}{|l|l|}
        \hline
        Input\#1                                        & Output\#1 \\ \hline
        \begin{tabular}[c]{@{}l@{}}2\\ 4 2\end{tabular} & 4.00 2.00 \\ \hline
    \end{tabular}
\end{table}
\subsubsection{提示与说明}
\paragraph{
    对于$100\%$的数据,满足$2\leq n\leq 10^5, 1\leq a_i \leq 10^6$
}
\newpage
\end{document}