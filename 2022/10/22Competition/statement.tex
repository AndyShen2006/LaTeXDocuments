\documentclass[final,11pt,oneside,UTF8]{report}
\usepackage{ctex}
\usepackage{float}
\usepackage{geometry}
\usepackage{graphicx}
\usepackage{amsmath,amsfonts,amssymb}
\usepackage{listings}
\usepackage{multicol}
\usepackage{multirow}
\begin{document}
\centerline{\LARGE{提高组模拟试题}}
\centerline{}
\centerline{\LARGE{比赛时间:2022年10月31日}}
\centerline{}
\centerline{出题人:}
\centerline{杭州第四中学\ 沈奕天(T1,T3,T4)}
\centerline{杭州高级中学\ 黄鹏程(T2)}
\centerline{}
\begin{table}[h]
    \centering
    \begin{tabular}{lllll}
        \hline
        \multicolumn{1}{|l|}{题目名称}    & \multicolumn{1}{l|}{桃花源的道路}   & \multicolumn{1}{l|}{扣篮大赛}          & \multicolumn{1}{l|}{花园}         & \multicolumn{1}{l|}{条件}            \\ \hline
        \multicolumn{1}{|l|}{题目类型}    & \multicolumn{1}{l|}{传统型}      & \multicolumn{1}{l|}{传统型}           & \multicolumn{1}{l|}{传统型}        & \multicolumn{1}{l|}{传统型}           \\ \hline
        \multicolumn{1}{|l|}{目录}      & \multicolumn{1}{l|}{road}     & \multicolumn{1}{l|}{basket}        & \multicolumn{1}{l|}{beauty}     & \multicolumn{1}{l|}{condition}     \\ \hline
        \multicolumn{1}{|l|}{可执行文件名}  & \multicolumn{1}{l|}{road}     & \multicolumn{1}{l|}{basket}        & \multicolumn{1}{l|}{beauty}     & \multicolumn{1}{l|}{condition}     \\ \hline
        \multicolumn{1}{|l|}{输入文件名}   & \multicolumn{1}{l|}{road.in}  & \multicolumn{1}{l|}{basket.in}     & \multicolumn{1}{l|}{beauty.in}  & \multicolumn{1}{l|}{condition.in}  \\ \hline
        \multicolumn{1}{|l|}{输出文件名}   & \multicolumn{1}{l|}{road.out} & \multicolumn{1}{l|}{basket.out}    & \multicolumn{1}{l|}{beauty.out} & \multicolumn{1}{l|}{condition.out} \\ \hline
        \multicolumn{1}{|l|}{每个测试点时限} & \multicolumn{1}{l|}{1.0秒}     & \multicolumn{1}{l|}{1.0秒}          & \multicolumn{1}{l|}{1.0秒}       & \multicolumn{1}{l|}{1.0秒}          \\ \hline
        \multicolumn{1}{|l|}{内存限制}    & \multicolumn{1}{l|}{512MiB}   & \multicolumn{1}{l|}{512MiB}        & \multicolumn{1}{l|}{512MiB}     & \multicolumn{1}{l|}{512MiB}        \\ \hline
        \multicolumn{1}{|l|}{子任务数目}   & \multicolumn{1}{l|}{20}       & \multicolumn{1}{l|}{10}            & \multicolumn{1}{l|}{20}         & \multicolumn{1}{l|}{20}            \\ \hline
        \multicolumn{1}{|l|}{测试点是否等分} & \multicolumn{1}{l|}{是}        & \multicolumn{1}{l|}{是}             & \multicolumn{1}{l|}{是}          & \multicolumn{1}{l|}{是}             \\ \hline
        提交源程序文件名                      &                               &                                    &                                 &                                    \\ \hline
        \multicolumn{1}{|l|}{对于C++语言} & \multicolumn{1}{l|}{road.cpp} & \multicolumn{1}{l|}{basket.cpp}    & \multicolumn{1}{l|}{beauty.cpp} & \multicolumn{1}{l|}{condition.cpp} \\ \hline
        编译选项                          &                               &                                    &                                 &                                    \\ \hline
        \multicolumn{1}{|l|}{对于C++语言} & \multicolumn{1}{l}{}          & \multicolumn{1}{l}{-O2 -std=c++14} & \multicolumn{1}{l}{}            & \multicolumn{1}{l|}{}              \\ \hline
    \end{tabular}
\end{table}
\newpage

\centerline{\LARGE{$\textbf{桃花源的道路}\text{(road)}$}}
\subsubsection{题目背景}
\paragraph{
    “林尽水源,便得一山,山有小口,仿佛若有光。便舍船,从口入。初极狭,才通人。复行数十步,豁然开朗。土地平旷,屋舍俨然,有良田、美池、桑竹之属。阡陌交通,鸡犬相闻。其中往来种作,男女衣着,悉如外人。黄发垂髫,并怡然自乐。”——陶渊明《桃花源记》
}
\paragraph{
    桃花源有很多阡陌(也就是田间小路),在桃花源中的居民有一天突然发现:有的阡陌不怎么好走了。原来,原来$2m$宽的路,在海陆变迁中毁坏了,有的变成了$1m$,有的变成了$3m$。现在,他们想要重新修一下这个路,使桃花源的交通重新通畅。
}
\subsubsection{题目描述}
\paragraph{
    现在,桃花源的居民给了你小路中其中一段的剖面图,用O表示$1m$路,用X表示$1m$障碍。他们可以将一个X移到一个O的地方上。你要编写一个程序,输出最少需要移动多少次,能够使得小路全部变到$2m$的状态。
}
\paragraph{桃花源的居民希望能够尽快通路,所以你的程序要尽可能快的解决这个问题。}
\subsubsection{输入格式}
\paragraph{一行一个字符串,用O和X描述了现在的道路情况。}
\subsubsection{输出格式}
\paragraph{一行一个整数$n$,表示最少需要移动的次数。}
\subsubsection{样例组}
\begin{table}[H]
    \begin{tabular}{|l|l|}
        \hline
        Input \#1: & Output \#1: \\ \hline
        XOXOOOXOOX & 1           \\ \hline
    \end{tabular}
\end{table}
\subsubsection{提示与说明}
\paragraph{样例解释:将第三位的X移到第四位即可。}
\paragraph{串的长度一定是$3n+1$,且$X$的个数是$n$个}
\paragraph{对于$100\%$的数据,保证字符串长度不超过$3 \times 10^6$}
\newpage
\centerline{\LARGE{$\textbf{扣篮大赛}\text{(basket)}$}}
\subsubsection{题目背景}
\paragraph{
    hpc 正在为今年的扣篮大赛作准备。
}
\subsubsection{题目描述}
\paragraph{
    hpc 腿伤刚康复,弹跳的能力有一定上限。扣篮大赛共有 $n$ 个篮筐,编号为 $1,2\cdots n$,高度分别为 $h_1,h_2\cdots h_n$,hpc 需要达到某个篮筐的高度才能在这个篮筐上扣篮。由于篮筐之间存在差异,扣篮的消耗也不尽相同,分别为 $s_1,s_2\cdots s_n$。hpc 要一次起跳并同时扣多个篮筐。他从起跳到落地的消耗为在空中经过的距离与扣篮消耗的总和。为了知道在自己的能力限度内最多能扣多少个篮筐,你需要求出他一次起跳扣 $1,2\cdots n$ 个篮筐的最少消耗。
}
\subsubsection{输入格式}
\paragraph{
    第一行一个整数 $n$。
}
\paragraph{
    第二行 $n$ 个整数,第 $i$ 个整数表示 $h_i$。
}
\paragraph{
    第三行 $n$ 个整数,第 $i$ 个整数表示 $s_i$。
}
\subsubsection{输出格式}
\paragraph{
    输出共 $n$ 行,每行一个整数。第 $i$ 行的整数表示扣 $i$ 个篮筐的最少消耗。
}
\subsubsection{样例组}
\begin{table}[H]
    \begin{tabular}{|l|l|}
        \hline
        Input \#1:                                                        & Output \#1:                                                   \\ \hline
        \begin{tabular}[c]{@{}l@{}}5\\ 1 2 3 4 5\\ 1 2 3 4 5\end{tabular} & \begin{tabular}[c]{@{}l@{}}3\\ 7\\ 12\\ 18\\ 25\end{tabular}  \\ \hline
        Input \#2:                                                        & Output \#2:                                                   \\ \hline
        \begin{tabular}[c]{@{}l@{}}5\\ 1 2 2 4 5\\ 5 4 3 4 1\end{tabular} & \begin{tabular}[c]{@{}l@{}}7\\ 11\\ 16\\ 22\\ 27\end{tabular} \\ \hline
    \end{tabular}
\end{table}
\subsubsection{提示与说明}
\paragraph{
    对于 $20\%$ 的数据,$n\le 10$。
}
\paragraph{
    对于 $60\%$ 的数据,$n\le 300$。
}
\paragraph{
    对于 $80\%$ 的数据,$n\le 4000$。
}
\paragraph{
    对于 $100\%$ 的数据,$2\le n\le 5000$,$1\le h_i,s_i\le 5\times 10^8$。
}
\newpage
\centerline{\LARGE{$\textbf{花园}\text{(beauty)}$}}
\subsubsection{题目背景}
\paragraph{
    Farmer John的农场里有一个花园,FJ正在为花园里种什么花而发愁……
}
\subsubsection{题目描述}
\paragraph{
    市场上有$m$种花,每朵花有$n$个“美观点”。其中,如果某一朵花和某一朵花有偶数个共同美观点,那么FJ就会认为美观的重复度过高而降低1的美观度,如果有奇数个共同美观点,就会增加1的美观度。(初始的美观度是0)
}
\paragraph{
    现在告诉你这些花的美观点,问最后总的美观度是多少。
}
\subsubsection{输入格式}
\paragraph{第一行三个整数$m,n$}
\paragraph{接下来$m$行,每行$n$个整数,代表了这个花具有的美观点数量。}
\subsubsection{输出格式}
\paragraph{
    一行一个整数,输出最后的美观度
}
\subsubsection{样例组}
\begin{table}[H]
    \begin{tabular}{|l|l|}
        \hline
        Input \#1:                                                                  & Output \#1: \\ \hline
        \begin{tabular}[c]{@{}l@{}}4 3\\ 1 2 4\\ 2 3 1\\ 4 3 2\\ 2 5 3\end{tabular} & -2          \\ \hline
    \end{tabular}
\end{table}
\subsubsection{提示}
\paragraph{
    对于$75\%$的数据,保证$2\le n\le 1\times 500$,$1\le m\le 10$
}
\paragraph{
    对于$100\%$的数据,保证$2\le n\le 1\times 10^4$,$1\le m\le 10$
}
\newpage
\centerline{\LARGE{$\textbf{条件}\text{(condition)}$}}
\subsubsection{题目描述}
\paragraph{在高中数学必修一中,有一课叫充分必要条件。其中,对于条件$p$和条件$q$,如果存在$p\rightarrow q$,就称$p$为$q$的充分条件,$q$为$p$的必要条件。如果存在$p \leftrightarrow q$,则称$p$为$q$的充分必要条件,$q$为$p$的充分必要条件(如果$p \leftrightarrow q$,则$p \rightarrow q$,$q \rightarrow p$)。显然的,这些条件之间存在着传递性,如果$a \rightarrow b$,$b \rightarrow c$,那么一定存在$a \rightarrow c$。另外,每个条件都是自己的充分必要条件。现在,给出若干条件和若干询问,问任意两个条件之间的关系。}
\subsubsection{输入格式}
\paragraph{第一行三个整数$n,m,q$,分别代表了条件数,关系数,询问数}
\paragraph{接下来$m$行,每行一个表达式,其中,表达式是这么定义的:整数 符号 整数}
\paragraph{例如:}
\paragraph{1 -> 2,说明1是2的充分条件(注意,箭头两边有空格,下同)}
\paragraph{2 <- 1,说明2是1的必要条件}
\paragraph{2 <-> 3,说明2是3的充分必要条件}
\paragraph{再接下来$q$行,每行两个整数,询问这两个整数之间的关系。}
\subsubsection{输出格式}
\paragraph{输出这些关系,表达形式与输入形式相同。(如果不存在任何关系,输出一行Fail)}
\subsubsection{样例组}
\begin{table}[H]
    \begin{tabular}{|l|l|}
        \hline
        Input \#1:                                                                                                                                & Output \#1:                                                                              \\ \hline
        \begin{tabular}[c]{@{}l@{}}4 4 2\\ 1 -\textgreater 2\\ 2 -\textgreater 3\\ 3 -\textgreater 1\\ 4 -\textgreater 2\\ 1 3\\ 4 3\end{tabular} & \begin{tabular}[c]{@{}l@{}}1 \textless{}-\textgreater 3\\ 4 -\textgreater 3\end{tabular} \\ \hline
    \end{tabular}
\end{table}
\subsubsection{提示与说明}
\paragraph{对于数据点1-15,保证$50\le n \le 500,100\le m \le 1000,1\le q \le 100$}
\paragraph{对于数据点16-20,保证$5000\le n \le 50000,10000\le m \le 100000,1\le q \le 500$,且对于每个点,都有至少100个点与它成充分必要关系。}
\newpage
\end{document}