\documentclass[UTF8,a4paper,oneside,final]{article}
\usepackage{ctex}
\usepackage{amsmath}
\usepackage{geometry}
\geometry{left=3.5cm,right=3.5cm,top=2.5cm,bottom=2.5cm}
\def\newpara{$\\\\\\\\\\$}
\author{AndyShen2006}
\title{Problems}
\date{7th.Nov.2020}
\begin{document}
\maketitle
\paragraph{题目后面有一个字母表达难度,E为简单,NE为较简单,N为正常,NH为较难,H为难}
\paragraph{一般情况下,E的题目以简单理解题为主,不会很难;NE的题目为正常证明题,
    N的题目为技巧题,NH的题目为较难证明题或较大的理解题,H的题目为难证明题}
\section{手求根号E}
\paragraph{
根号数是一种特殊的无理数,它无法得到准确值,有时我们需要他的近似形式,
就要手求根号.\\
小华想要求出$\sqrt{2}$的近似值,因为
$1<\sqrt{2} <2$,所以他将[1,2]这一段十等分,易得
$$1.1^2 <1.2^2 <1.3^2 <1.4^2 <2<1.5^2 <\cdots <1.9^2$$\\
故$\sqrt{2}$在$[1.4,1.5]$这一段内.
}
\paragraph{以此类推,请你分别用这种方法求出$\sqrt{5} ,\sqrt{7}$的值}
\newpara
\section{平均数NE}
\paragraph{设$a,b,c$的平均数为$M;a,b$的平均数为
    $N;N,c$的平均数为$P.$,若$a>b>c,$则$M$与$P$
    的大小关系是$\_\_\_\_\_\_\_\_$
}
\newpage
%$\\\\$
%\section{n星瓢虫}
%\paragraph{
%    小明和小雪抓了一些瓢虫,小雪抓的瓢虫上面的
%    斑点数之和是小明抓的瓢虫的斑点数之和的$13$倍.
%    当小雪将自己斑点数最少的一个瓢虫送给小明时
%    她的瓢虫斑点数之和就只是小明的瓢虫斑点数之
%    和的$8$倍了,试证:小雪采了不多于$23$个瓢虫.
%}
\section{神奇的不等式NH}
\paragraph{若$\frac{a}{b} <\frac{c}{d}$ ,且$bd$为正数,试证
    $$\frac{a}{b} <\frac{a+c}{b+d} <\frac{c}{d}$$}
%\paragraph{提示:要用因式分解}
\newpara
\section{正负乐何极H}
\paragraph{实数$a,b,c$满足不等式
    $|a|\geq |b+c|,|b|\geq |c+a|,|c|\geq |a+b|.$求证:
    $$a+b+c=0$$}
\newpara
\section{松松紧紧N}
\paragraph{求出$\cfrac{1}{\cfrac{1}{12001}+\cfrac{1}{12002}+\cfrac{1}{12003}+\cdots +\cfrac{1}{13000}}$的整数部分}
\newpara
\section{相对与绝对NE}
\paragraph{设a,b,c为实数,试证$$|a-b|\leq |a-c|+|c-b|$$}
\newpara
\section{算数压缩NH}
\end{document}