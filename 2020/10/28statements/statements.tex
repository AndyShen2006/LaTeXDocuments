\documentclass[final,11pt,oneside,UTF8]{report}
\usepackage{ctex}
\usepackage{float}
\usepackage{geometry}
\usepackage{graphicx}
\usepackage{amsmath,amsfonts,amssymb}
\title{Statements}
\author{AndyShen2006}
\date{Oct.28th.2020}
\begin{document}
\maketitle
由于写的有点匆忙,所以没画图,图就自己看吧。
\subsubsection{T1(圣彼得堡1994年八年级题)}
\paragraph{
    这题十分简单,简单过一下,大概看一下过程。D点是AB中点
}
\paragraph{
    $$\because \angle DAO=\angle OAE,BE\perp AC,OD\perp AB,DO=OE$$
    $$\therefore \triangle AOD\cong \triangle AOE(AAS)$$
    $$\therefore AD=DB=AE$$
    $$\therefore AB=2AE$$
    $$\therefore \angle ABE=30^\circ ,\angle A=60^\circ$$
}
\subsubsection{T2(1990年北京市竞赛题)}
\paragraph{
    这题稍有些难度,但是不算很难。这道题目中要证明BD平分,最好的办法是根据
    等腰三角形三线合一,构造等腰三角形去做,所以延长AE,BC至点F。
}
\paragraph{
    $$\angle FAC=90^\circ -\angle AFB=\angle CBD$$
    $$\because AC=BC,\angle ACF=\angle DCB=90^\circ$$
    $$\therefore \triangle AFC\cong \triangle BDC(ASA)$$
    $$\because AE=\frac{1}{2}BD=\frac{1}{2}AF$$
    $$\therefore AE=EF$$
    \begin{center}
        又$BE\perp AF,EB=EB$
    \end{center}
    $$\therefore AB=AF$$
    \begin{center}
        BD平分$\angle FBA$
    \end{center}
}
\subsubsection{T3(1998年全国初中数学竞赛试题)}
\paragraph{
    出了这题是我的失误Orz,因为这题显著的要用中位线。该中位线便是ED
}
\paragraph{
    我们预习一下中位线的定义和性质,中位线是三角形两条边的中点的连线(定义)
    ,中位线的长度是底边的一半,中位线与底边平行(性质)\\
    证明方法是延长中位线至二倍,连接底边的点和中位线的点,SAS证明与原三角形全等。
    接着用两条对边相等且平行(还是SAS证明)证明二倍中位线和底边相等且平行,即证
}
\paragraph{
    这题连接ED,这是一条中位线,由于两条中线垂直,则这个四边形面积是两中线乘积除以二
    $(S_DEBC =4*6/2=12)$(设一条是底边,分成两个三角形,然后分别乘以它们的高,分配律一下就得到了)
    因为ED是中位线$,$所以$S_\triangle AED=S_\triangle BED=S,S_\triangle DBC=2S_\triangle DEB=2S$
    从而可以得到$S_\triangle ABC=4S=\frac{4}{3}*S_DEBC=16$
}
\subsubsection{T4(第20届全俄奥林匹克题)}
\paragraph{
    又是一道简单题QAQ,这题目就比作业本的题目难一点。
}
\paragraph{
    $$\because OA=OC,\angle C_1 OA=\angle A_1 OC,\angle AC_1 O=\angle OA_1 C=90^\circ$$
    $$\therefore \triangle AOC_1 \cong \triangle COA_1 ,\angle OAC=\angle OCA$$
    $$\therefore \angle OAC_1 =\angle OCA_1$$
    $$\therefore \angle CAC_1 = \angle ACA_1$$
    $$\therefore AB=BC$$
    \begin{center}
        即$\triangle ABC$是等腰三角形
    \end{center}
}
\end{document}